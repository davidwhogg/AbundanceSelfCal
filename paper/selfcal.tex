% This file is part of the AbundanceSelfCal project
% Copyright 2019, 2020 the authors.

\documentclass[12pt, letterpaper]{article}

%% text macros
\newcommand{\project}[1]{\textsl{#1}}
\newcommand{\acronym}[1]{{\small{#1}}}
\newcommand{\apogee}{\project{\acronym{APOGEE}}}
\newcommand{\sdssiv}{\project{\acronym{SDSS-IV}}}
\newcommand{\dr}{\acronym{DR17}}

%% math macros
\newcommand{\teff}{{T_{\mathrm{eff}}}}
\newcommand{\logg}{\log g}

\begin{document}\sloppy\sloppypar\raggedbottom\frenchspacing

\section*{\raggedright
  A data-driven causal model to recalibrate element-abundance measurements,
  and to infer original birth abundances}
\noindent
ACE, DWH, MKN, HWR, and others

\paragraph{Abstract:}
There are multiple large-scale spectroscopic surveys of stars in the
Milky Way, currently comprising a global total of millions of stars with
detailed element-abundance information.
Many millions more are expected in the coming decade.
Element-abundance measurements are of great scientific importance because
surface abundances are expected to be (almost) invariant over a stellar
lifetime and thus provide information about the relationships between
the detailed history of star formation and the current configuration of the Galaxy.
This scientific program is fraught, however:
The element-abundance inferences are strongly model-dependent, being
affected by the treatment of computationally challenging physics, such
as non-eqilibrium and stochastic time-dependent effects.
And the surface abundances can truly change over cosmic time, because
of dredge-up, accretion, or other stellar-evolution effects.
That is, for both model mis-specification and stellar-evolution
reasons, the empirical element abundances in stellar parameter
catalogs show strong variations with stellar evolutionary phase, even
conditioning on position and velocity in the Galaxy.
Here we present a self-calibration of the stellar element abundances for
luminous red-giant stars in the \sdssiv--\apogee\ \dr\ Catalog [correct?].
This self-calibration is based on a particular hierarchical model
structure, or particular set of assumptions about the possible
existence (and enforced non-existence) of causal relationships between
birth abundances and properties of the stars, including position,
velocity, surface gravity, and observing details (such as which instrument
or spectroscopic fiber was employed).
Up to some overall degeneracies, the model produces a re-calibrated
set of element abundances for each star, which constitute an inference
of what the element abundances \emph{would have been at the time of
  the star's birth}.
[Statements about performance and validation.]

\section{Introduction}

We don't know how consistent our abundance measurements are as we scan
around the Milky Way...

And now there aren't just abundance measurements.
There are ages and...
All of these are uncertain, and have complex calibration issues...

The best current tests make use of globular and open clusters.
We ourselves are guilty of using such tests...

Importantly, abundances are only expected or permitted to depend on certain
things.
That is, one can take a \emph{causal} approach to this problem...

We can't distinguish between systematic errors and incorrectness of
our assumptions. But we don't need to!

Relatedly, the best calibration systems currently operating on astronomical
data are self-calibrations.
These systems make use of causal structure (and large data sets) to
separate real from spurious trends in, say, photometry of stars or
temperatures of cosmic background pixels.
The general idea is of very wide applicability...

\section{Method}

Imagine that we have a catalog of $N$ stars.
Each star $n$ has abundance measurements $y_n$.
That is, for each star $n$, we have assembled a set of abundance ratio
measurements into a data vector (or blob) $y_n$.
We believe that the true abundance ratios $z_n$ of the star are
closely related to, but not identical to, the measured abundance
ratios $y_n$.
We are going to use things we know about the causal relationships
between the abundances and other data to infer the true abundances
$z_n$ from the observed abundances $y_n$.

For each star $n$ we have, in addition to the abundance measurements
$y_n$ two other pieces of information.
The first is the position $X_n$ of the star in phase space; that is,
the position and velocity of the star in the Galaxy.
The second is meta-data about the star.
This can be thought of as a vector or blob $x_n$ of things we know
about the star like its apparent magnitude $J_n$, temperature
$\teff_n$, and surface gravity $\logg_n$.
The meta-data $x_n$ also includes things about how the star was
observed, like which spectrograph was used to observe it, which fiber
within that spectrograph, what time of year, the airmass of the
observation, and so on.

The causal argument here is that the true abundances $z_n$ of the star
will depend only on the phase-space position of the star $X_n$, but not at
all on the meta-data $x_n$.
That is...
...

In what sense are these assumptions wrong...?

\paragraph{Acknowledgements:}
Thanks to Jo Bovy (Toronto) for valuable input.

\end{document}
