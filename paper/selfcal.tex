% This file is part of the AbundanceSelfCal project
% Copyright 2019 the authors.

\documentclass[12pt, letterpaper]{article}

%% math macros
\newcommand{\teff}{{T_{\mathrm{eff}}}}
\newcommand{\logg}{\log g}

\begin{document}\sloppy\sloppypar\raggedbottom\frenchspacing

\section*{Self-calibration of stellar element abundances}
\noindent
\textbf{David W. Hogg} (NYU, MPIA, Flatiron), and \textbf{others}

\section{Introduction}
We don't know how consistent our abundance measurements are as we scan
around the Milky Way...

And now there aren't just abundance measurements.
There are ages and...
All of these are uncertain, and have complex calibration issues...

The best current tests make use of globular and open clusters.
We ourselves are guilty of using such tests...

Importantly, abundances are only expected or permitted to depend on certain
things.
That is, one can take a \emph{causal} approach to this problem...

Relatedly, the best calibration systems currently operating on astronomical
data are self-calibrations.
These systems make use of causal structure (and large data sets) to
separate real from spurious trends in, say, photometry of stars or
temperatures of cosmic background pixels.
The general idea is of very wide applicability...

\section{Method}
Imagine that we have a catalog of $N$ stars.
Each star $n$ has abundance measurements $y_n$.
That is, for each star $n$, we have assembled a set of abundance ratio
measurements into a data vector (or blob) $y_n$.
We believe that the true abundance ratios $z_n$ of the star are
closely related to, but not identical to, the measured abundance
ratios $y_n$.
We are going to use things we know about the causal relationships
between the abundances and other data to infer the true abundances
$z_n$ from the observed abundances $y_n$.

For each star $n$ we have, in addition to the abundance measurements
$y_n$ two other pieces of information.
The first is the position $X_n$ of the star in phase space; that is,
the position and velocity of the star in the Galaxy.
The second is meta-data about the star.
This can be thought of as a vector or blob $x_n$ of things we know
about the star like its apparent magnitude $J_n$, temperature
$\teff_n$, and surface gravity $\logg_n$.
The meta-data $x_n$ also includes things about how the star was
observed, like which spectrograph was used to observe it, which fiber
within that spectrograph, what time of year, the airmass of the
observation, and so on.

The causal argument here is that the true abundances $z_n$ of the star
will depend only on the phase-space position of the star $X_n$, but not at
all on the meta-data $x_n$.
That is...
...

In what sense are these assumptions wrong...?

\end{document}
