% This file is part of the AbundanceSelfCal project
% Copyright 2020 the authors.

\documentclass[12pt, letterpaper]{article}

%% page layout
\addtolength{\topmargin}{-1.00in}
\addtolength{\textheight}{2.00in}

%% text macros
\newcommand{\project}[1]{\textsl{#1}}
\newcommand{\acronym}[1]{{\small{#1}}}
\newcommand{\apogee}{\project{\acronym{APOGEE}}}
\newcommand{\sdssiv}{\project{\acronym{SDSS-IV}}}
\newcommand{\dr}{\acronym{DR17}}

%% math macros
\newcommand{\teff}{{T_{\mathrm{eff}}}}
\newcommand{\logg}{\log g}

\begin{document}\sloppy\sloppypar\raggedbottom\frenchspacing

\section*{\raggedright
  Abundance maps of the Milky Way disk:
  Accounting for systematic and evolutionary effects on the red-giant branch
  }
\noindent
ACE, DWH, MKN, HWR

\paragraph{Abstract:}
There are conflicting claims in the literature about the element
abundances of stars in the Galactic bulge and bar, as compared to the
inner parts of the disk.
One source of confusion is that, because selection effects are usually
a function of angular stellar density, and because luminosity changes
as stars ascend the red-giant branch, there are no samples of stars
for which the same range of metallicities and evolutionary states is
visible at all Galactocentric radii.
Here we use stars from the upper red-giant branch---stars more
luminous than the red clump---as observed by \apogee, to look at the
element-abundance gradients in the Milky Way disk, in various
abundance ratios.
Although the gradients are clearly measurable in the raw
\apogee\ measurements, we improve precision by simultaneously fitting
for the gradients and dependences of the measurements on red-giant
evolutionary phase.
This evolutionary-phase model accounts for both systematics in the
abundance measurements, and evolutionary effects in red-giant surface
abundances.
We find that elements XX and YY have measurements that are biased
systematically as a function of evolutionary phase in the \apogee data.
Whether or not we apply our evolutionary-phase model,
we find that all mean abundance gradients vary smoothly and
monotonically with Galactocentric radius from 0 to 20\,kpc.
We find no evidence for a distinct stellar population that formed the
bar or bulge; everything we find is consistent with secular (correct
words here?) creation of the Milky Way's bulge.

\end{document}
