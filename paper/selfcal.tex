% This file is part of the AbundanceSelfCal project
% Copyright 2019 the authors.

\documentclass[12pt, letterpaper]{article}

\begin{document}

\section*{Self-calibration of stellar element abundances}
\noindent
\textbf{David W. Hogg} (NYU, MPIA, Flatiron), and \textbf{others}

\section{Introduction}
We don't know how consistent our abundance measurements are as we scan
around the Milky Way...

And now there aren't just abundance measurements.
There are ages and...
All of these are uncertain, and have complex calibration issues...

The best current tests make use of globular and open clusters.
We ourselves are guilty of using such tests...

Importantly, abundances are only expected or permitted to depend on certain
things.
That is, one can take a \emph{causal} approach to this problem...

Relatedly, the best calibration systems currently operating on astronomical
data are self-calibrations.
These systems make use of causal structure (and large data sets) to
separate real from spurious trends in, say, photometry of stars or
temperatures of cosmic background pixels.
The general idea is of very wide applicability...

\section{Method}
Imagine that we have a catalog of $N$ stars...

\end{document}
